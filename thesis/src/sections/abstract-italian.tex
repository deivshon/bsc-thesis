\begin{center}
  \textsc{Sommario}
\end{center}
\noindent

I contenuti sul web sono diventati progressivamente più complessi negli anni: da pagine completamente statiche a vere e proprie applicazioni interattive. A rendere possibile l'interattività è JavaScript: un linguaggio dinamicamente tipato. Con il tempo, lo sviluppo di applicazioni su scala ha mostrato le limitazioni dell'approccio del linguaggio, e molte soluzioni sono emerse che mirano a dare agli sviluppatori web un'esperienza più ergonomica e sicura.

Elm è nato in questo contesto: un linguaggio funzionale puro, staticamente tipato che include supporto ad una architettura model-view-update (conosciuta come The Elm Architecture) che permette agli sviluppatori di concentrarsi sullo sviluppo della logica applicativa.

Questa tesi mira ad esplorare le caratteristiche ed i trade-off di Elm, investigando come influenzano lo sviluppo di applicazioni web e comparandolo con alcune delle più usate soluzioni nate successivamente.
