\chapter*{Introduction}
% Functional languages for frontend, mention purescript/rescript
\addcontentsline{toc}{chapter}{Introduction}
\setcounter{page}{1}

JavaScript is the web's scripting language, and while some forms of interactivity such as form submissions are possible without it, less than 2\% of websistes don't include any \cite{noauthor_usage_nodate}. JavaScript is an imperative, weakly and dynamically typed programming language. While it can be used directly, today it is also a transpilation target of a plethora of programming languages \cite{noauthor_list_nodate}.

Most modern websites are created with frameworks that facilitate the creation of dynamic UIs by letting the developer write declarative code and handling reconciliation between state and the DOM internally. While frameworks solve a class of problems, they can't solve JavaScript's: not having a static type system makes it easy for subtle bugs to go undetected, and makes refactoring existing code hard without introducing regressions.

For this reason, many of the languages that transpile to JavaScript include static type systems. TypeScript is the most successful one \cite{noauthor_most_nodate}, it seeks to add only a thin layer on top of plain JavaScript that allows for a structural type system to be used, and provides seamless interoperation. TypeScript's type system is not sound, and making it sound is not a goal of the language \cite{noauthor_typescript_nodate}, which also includes escape hatches such as type assertions, that when abused have the potential to cause runtime errors.

Other languages such as PureScript, ReScript and Elm take a different approach, abstracting more from plain JavaScript in order to not be bound by its inherent limitations.

\chapter{Elm}
Elm is a purely functional programming language that transpiles to JavaScript. It aims to eliminate all possible runtime errors and let developers write declarative code within a model-view-update framework, known as The Elm Architecture.

\section{Stated design goals}
% No runtime errors, even at the cost of limited interoperability

\subsubsection{No runtime errors}
If an Elm program compiles and has been type checked, it is almost completely guaranteed to not crash. This is achieved through its strong static type system and limited interoperability: unlike in similar languages, it's been restricted to message passing and Web Components usage only, since calling foreign functions could allow malformed data to sneak into the Elm runtime.

\subsubsection{Predictability}
Side effects are a significant source of impredictability. Being a purely functional language, the developer knows when they call a function in Elm it's always a pure function, which means that no side effects can happen. Data can only be transformed and, if needed, returned to the caller.

\subsubsection{Simplicity}
Elm is meant to be easy to learn and use. This is the reason the language itself is not as comprehensive as some of its purely functional competitors, not providing advanced features such as higher-kinded types and type classes.

\section{Language features}
% ADTs, pattern matching, strong inference
\subsubsection{Base types}
Elm provides the following primitive types:
\begin{itemize}
    \item \texttt{String}
    \item \texttt{Int}
    \item \texttt{Float}
    \item \texttt{Bool}
\end{itemize}
The generic \texttt{List} type is also available to create lists of a given type.\\

\subsubsection{Type annotations and inference}
Elm supports type annotations, these help clarify intent and allow readers to understand the function signature by just reading the code, instead of relying on IDE tools or needing to go through the implementation:
\begin{minted}{elm}
addToListItems : Int -> List Int -> List Int
addToListItems n ls =
    List.map (\x -> x + n) ls
\end{minted}

Elm's (unofficial) LSP supports a ``add type annotation'' action. Using it on the unannotated version of the function yields a slightly different result:
\begin{minted}{elm}
addToListItems : number -> List number -> List number
addToListItems n ls =
    List.map (\x -> x + n) ls
\end{minted}
\texttt{number} is neither a \texttt{Float} nor an \texttt{Int} type, instead, it's a type that accepts both. When adding a type annotation using the LSP's action, the signature added is the same one that is computed when Elm performs type inference, which is the widest the implementation supports.\\
In fact, both \texttt{Float}s and \texttt{Int}s can be passed as the \texttt{addToListItems} function's first parameter, since the \texttt{+} operator in Elm supports both.

\subsubsection{Records}
Elm supports records, single types containing multiple named components. Without records, if we needed to write a function that renders a \textit{Todo} item, we would have to write it like this:
\begin{minted}{elm}
viewTodo id content createdAt updatedAt done =
    ...
\end{minted}
When types in the application domain grow significantly, writing all components separately becomes cumbersome. Instead, we can use a record, and access its components with the dot notation:
\begin{minted}{elm}
type alias Todo =
    { id : String
    , content : String
    , createdAt : Int
    , updatedAt : Int
    , done : Bool
    }

viewTodo todo =
    ...

isLongTodo todo =
    String.length todo.content > 50
\end{minted}

\subsubsection{Sum types}
Sum types, commonly referred to within the Elm community as \textit{custom types} \cite{noauthor_custom_nodate}, are also supported, increasing the type system's expressiveness.\\

They can be used to create regular enums:
\begin{minted}{elm}
type PackageDeliveryStatus
    = InWarehouse
    | InProgress
    | Delivered
\end{minted}

But the types specified may also be associated with an arbitrary number of other types. For instance, we may want to specify when the \texttt{InProgress} and \texttt{Delivered} statuses were reached using a timestamp, and the final delivery address:
\begin{minted}{elm}
type PackageDeliveryStatus
    = InWarehouse
    | InProgress Int
    | Delivered Int String
\end{minted}

An instance of a sum type can be accessed using the \texttt{case/of} pattern matching syntax:
\begin{minted}{elm}
isOutOfWarehouse status =
    case status of
        InWarehouse ->
            False

        _ ->
            True
\end{minted}

The \texttt{\_} character matches all remaining possibilities. Since Elm's pattern matching is required to be exhaustive, leaving any option not covered by a branch results in a compile time error. This ensures the developer doesn't leave any possible value unhandled.\\

Elm's pattern matching also supports accessing the values associated with the sum type instance:
\begin{minted}{elm}
getStatusString status =
    case status of
        InWarehouse ->
            "In warehouse"

        InProgress ts ->
            "In progress since epoch " ++ String.fromInt ts

        Delivered ts address ->
            "Delivered to "
                ++ address
                ++ " at epoch "
                ++ String.fromInt ts
\end{minted}

\subsubsection{Error handling}
Elm has opted to represent errors as values. A built-in \texttt{Result} type is present
\begin{minted}{elm}
type Result e v
    = Ok v
    | Err e
\end{minted}

Being generic, any type can be used for errors. For instance, another sum type may be used as the \texttt{e} error type, enumerating the possible failure modes:
\begin{minted}{elm}
type DivisionError
    = DivisionByZero


safeDivide : Float -> Float -> Result DivisionError Float
safeDivide a b =
    if b == 0 then
        Err DivisionByZero

    else
        Ok (a / b)
\end{minted}
This powerful pattern allows us to make sure any function caller will not be able to use an invalid result: they must first pattern match on it, and, in case of an error, they know exactly what went wrong.\\

Similarly, a \texttt{Maybe} type is available, and can be used wherever a value may or may not be present:
\begin{minted}{elm}
type Maybe a
    = Just a
    | Nothing

type alias Person =
    { name : String
    , age : Int
    , email : Maybe String
    }
\end{minted}

\section{The Elm Architecture (TEA)}
% Mention impossibility of tying messages with a particular state, which often result in boilerplate (e.g. what do you do if you get a message that prompts you to fetch stuff but you don't have a token loaded? You need to handle that case for all messages if you have some token in your state. Solution is to pass the token in the message but this means you have to drill the token down to all components that use it)

The standard way to build interactive web applications with Elm is to use its built-in, declarative, model-view-update framework, known as The Elm Architecture.

\subsubsection{Model}
An Elm application must have a model that will be used in the view/update cycle. Said model contains the state of the application. For instance, a sample application containing a single checkbox would come with a single \texttt{Bool} as its state, representing whether the checkbox is checked or unchecked:

\begin{minted}{elm}
type alias Model = Bool
\end{minted}

\subsubsection{Update}
In order to change the state when the user performs an action we need to define a message type. This is conventionally named \texttt{Msg} and is usually a sum type. In our simple checkbox example application, the only action to associate to a message is the toggling of the checkbox:
\begin{minted}{elm}
type Msg
    = CheckboxClicked
\end{minted}

We should then define a function that takes a message and the current state as parameters, and returns the new state. This function is conventionally named \texttt{update}:
\begin{minted}{elm}
update : Msg -> Model -> Model
update msg model =
    case msg of
        CheckboxClicked ->
            not model
\end{minted}

In more complex applications, the update function may not only return the state, but a tuple containing the state and a command to execute, in turn associated with a message to be handled once it completes:
\begin{minted}{elm}
update : Msg -> Model -> ( Model, Cmd Msg )
update msg model =
    [...]
\end{minted}

\subsubsection{View}
The view function uses the current state to render HTML based on it. Elm comes with a \texttt{Html} package exposing what's necessary to work with all tags, attributes and events.\\
This is a possible view function for the simple checkbox app:
\begin{minted}{elm}
import Html as H
import Html.Attributes exposing (checked, type_)
import Html.Events exposing (onClick)

[...]

view : Model -> H.Html Msg
view model =
    H.div []
        [ H.input
            [ type_ "checkbox"
            , checked model
            , onClick CheckboxClicked
            ]
            []
        , H.span []
            [ H.text
                (if model then
                    "Checked"

                 else
                    "Unchecked"
                )
            ]
        ]
\end{minted}
In the first parameter of a Elm's function for an HTML tag, all attributes and events related to the node are passed as a list, while the second parameter is a list containing its children nodes.\\

A careful look at the \texttt{view} function signature shows the \texttt{H.Html} type is generic and associated to the \texttt{Msg} type. This is because the rendered HTML will only send messages of the \texttt{Msg} type, ensuring the \texttt{update} function can handle them properly.

\subsubsection{Initialization}
An initial state must also be provided, conventionally named \texttt{init}.\\
To bring it all together, update, view and initialization can then be passed to the \texttt{Browser.sandbox} function, which takes care of starting and running the Elm runtime:
\begin{minted}{elm}
init : Model
init =
    False

[...]

main : Program () Model Msg
main =
    Browser.sandbox
        { init = init
        , update = update
        , view = view
        }
\end{minted}

The update function runs every time a message is sent, and the state it returns replaces the current one. At this point, the DOM is updated according to what the view function called with the new state returns.

\subsubsection{Subscriptions}

\texttt{Browser.sandbox} is not the only way to initialize an Elm application, other functions such as \texttt{Browser.element} and \texttt{Browser.application} can be used, both of which support the \texttt{subscriptions} field.\\

Elm subscriptions allow the Elm runtime to listen to events. For instance, subscriptions can be used to listen to periodic ticks using the built-in \texttt{Time} library, or, more commonly, to listen to inbound messages from JavaScript when interoperating with message passing. In this context, each type of message that can be received from JavaScript is associated with an internal message that can be matched in the \texttt{update} function.

\section{Internals}
When an Elm program is transpiled it produces a single-page web application (SPA). This kind of application avoids fully loading a new page when clicking on internal links. Instead, it directly modifies the DOM to display the page clicked, performing only the necessary API calls, if any.\\

To improve user-perceived performance in a SPA, two critical factors are:
\begin{itemize}
    \item Computational overhead (compared to plain JavaScript)
    \item Asset size
\end{itemize}

\subsection{Overhead}
A fundamental feature of the vast majority of frontend frameworks today, Elm included, is the automatic synchronization between internal state and the DOM. This is the process that creates overhead compared to using manual DOM manipulations with plain JavaScript.\\

At runtime, Elm uses a Virtual DOM (often abbreviated to VDOM) to keep track of the real DOM. The Elm VDOM is a lightweight representation of the DOM in JavaScript, which supports faster operations compared to those that would be possible using the real DOM. \cite{noauthor_htmllazy_nodate}\\

Whenever a new model is returned from them main \texttt{update} function, Elm compares the current VDOM with the next one computed using the new state, so that it can apply the real and costly DOM operations only on the elements that actually need it. This comparison process is usually called \textit{diffing}, and it is executed by comparing each node of previous VDOM with its current counterpart.\\

There's also a manual, opt-in way to optimize the diffing process: \texttt{Html.Lazy}. Let's take an example function that produces a big tree of HTML nodes:
\begin{minted}{elm}
viewComplexData : String -> Int -> H.Html msg
viewComplexData content amount =
    H.div []
        [ [...] -- Complex rendering logic 
                -- with many children and subchildren
        ]
\end{minted}

Normally, we would use it like this:
\begin{minted}{elm}
view data =
    H.div []
        [ [...]
        , viewComplexData data.content data.amount
        , [...]
        ]
\end{minted}

But we may also wrap the \texttt{viewComplexData} function call with \texttt{lazy2}\footnote[1]{the 2 at the end simply signifies that it is the \texttt{lazy} function whose wrapped view function takes two arguments. This is necessary because Elm does not support variadic functions.}:
\begin{minted}{elm}
import Html.Lazy exposing (lazy2)
[...]
view data =
    H.div []
        [ [...]
        , lazy2 viewComplexData data.content data.amount
        , [...]
        ]
\end{minted}


Since Elm is a purely functional language, the compiler is guaranteed that all functions in a program are referentially transparent (i.e. given the same input they always produce the same output). During the diffing process, if the Elm runtime encounters a normal node, it proceeds to compare its children nodes, but if the node was created using \texttt{lazy}, the costly part of the comparison is avoided: Elm can simply check that the function \texttt{lazy} is wrapping and its arguments are the same, with the formal guarantee that, if this is true, then nothing can have changed in the output nodes.

\subsection{Asset size}
For all frontend solutions asset size is an important metric, especially for those that produce a SPA: before the user can see anything on the screen, the JavaScript asset that contains the code to render the HTML must be fetched, parsed and executed.

The default \texttt{elm make} command without flags produces unoptimized JavaScript code, while the \texttt{--optimize} flag, which is suggested for production builds, tells the Elm compiler to perform various optimizations that, among other things, help make the final asset size smaller, such as dead code elimination. The resulting code is not minified, but any JavaScript minifier can be used independently. \cite{noauthor_minification_nodate}

Some optimizations that would normally not be applicable at all or, at least, not without significant risks, can be safely used on the transpiled code:
\begin{itemize}
    \item \textbf{Record fields shortening}\\In JavaScript object fields can be accessed dynamically (i.e. not using the field name explicitly but using a variable that contains it), leading to possible runtime crashes if they are changed for optimization purposes. Elm does not support dynamic fields access, making it possible to safely shorten the record fields at build time.
    \item \textbf{Unused function calls elimination}\\If the result of a function call is not used, that call and, if present, assignment, can be completely eliminated. This would not be safe if Elm didn't guarantee that functions don't have side effects, since in such a case it would be impossible to tell if the function modifies values that are accessed somewhere else.
\end{itemize}
