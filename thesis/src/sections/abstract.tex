\begin{center}
  \textsc{Abstract}
\end{center}
\noindent

Content served on the web has become increasingly complex over the years: from completely static pages to full-fledged interactive applications. Powering all of the interactivity is JavaScript, a dynamically typed, multi-paradigm programming language. Over time, building applications at scale showed the limitations of the language's approach, and many solutions emerged that aimed to provide a more ergonomic and safer experience for web developers.

Elm was designed in this context: a purely functional, statically typed language that includes built-in support for a model-view-update architecture (known as The Elm Architecture) that allows developers to focus on the application logic.

This thesis aims to explore the Elm language features and trade-offs, investigating how they influence building web applications and how it compares to some of the most popular frontend solutions today.
