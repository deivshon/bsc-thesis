\begin{center}
  \textsc{Abstract}
\end{center}
\noindent

Content served on the web has become increasingly complex over the years: from completely static pages to full-fledged interactive applications. Powering all of the interactivity is JavaScript, a weakly, dynamically typed, multi-paradigm programming language. Over time, building applications at scale showed its limitations, and many solutions emerged that aimed to provide a more ergonomic and safer experience for web developers.

Elm was designed in this context: a purely functional, strongly and statically typed language that includes built-in support for a model-view-update architecture (known as The Elm Architecture) that allows developers to focus on the application logic instead of manual reconciliation between state and the DOM.

This thesis aims to explore Elm language features and trade-offs, investigating how they influence building web applications and how it compares to some of the most popular frontend solutions today.
